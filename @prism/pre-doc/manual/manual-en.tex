% !TEX TS-program = lualatex

\documentclass{tutodoc}

\usepackage{../../contrib/translate/en/manual/preamble.cfg}

\usepackage[subpreambles = true]{standalone}

\makeatletter
\renewcommand{\minted@cachedir}{_minted-xtra-cache}
\makeatother

\begin{document}

\title{\thisproj{} project}
\author{Christophe BAL}
\date{11 Oct 2025 - Version 1.1.0}

\maketitle

\begin{abstract}
    The \thisproj{} project\,%
    \footnote{
        The name comes from \tdocquote{%
        \tdocprewhy{@.esthetic} \tdocprewhy{P.roducts} 
        for
        \tdocprewhy{R.epresenting} \tdocprewhy{I.nformative}
        \tdocprewhy{S.cientific} \tdocprewhy{M.aps}}.
        %
        This name is a double play on words:\kern3pt%
        \begin{enumerate*}[label={[\arabic*]}]
            \item a prism is where light is split into an informative spectrum, symbolizing how data or visuals are decomposed into meaningful color and style,
            and
            \item where light meets the prism, it breaks down into an informative spectrum ("@" can be read "at").
        \end{enumerate*}
    }
    provides small size color palettes that can be used to create expressive color maps for graphics in different contexts.

%    \smallskip
%
%    \noindent
%    \emph{\textbf{Remark :} this documentation is also available in <<DOC-LANGS>>.}
\end{abstract}

\tdocsep

{
\small

\bgroup
    \addtokomafont{subsection}{\centering}
    \subsection*{Last changes}
\egroup

\begin{tdocbreak}[version = 1.1.0, date = 2025-10-14]
    \item Duplicate palettes and those that are reverse of others are ignored (strict equalities only).
\end{tdocbreak}


\begin{tdocnew}
    \item New palettes added: \verb+BurningGrass+, \verb+GeoRainbow+ and \verb+PastelRainbow+ (\tdocpack{luadraw} package creation process used).

    \item The \tdocpack{luadraw} palette product has a new dictionary like variable \tdoccodein{lua}{getPal} to access a palette using its name (as a string variable).
\end{tdocnew}


\begin{tdocupdate}
    \item Palette contributions: in the mandatory \verb+extend.py+ file, \tdoccodein{py}{build_code} must work with the dictionary of all the palettes, and manage a credit to the \thisproj\ project.
\end{tdocupdate}
}

\newpage
\tableofcontents
\newpage

\subimport{../../contrib/translate/en/manual/to-know/}{motivations.tex}
\subimport{../../contrib/translate/en/manual/to-know/}{backstage.tex}
\subimport{../../contrib/translate/en/manual/to-know/}{reuse-from.tex}
\subimport{../../contrib/translate/en/manual/to-know/}{searching.tex}
\subimport{../../contrib/translate/en/manual/products/}{preamble.tex}
\subimport{../../contrib/translate/en/manual/products/}{luadraw.tex}
\subimport{../../contrib/translate/en/manual/contrib-how-to/}{prologue.tex}
\subimport{../../contrib/translate/en/manual/contrib-how-to/}{translate.tex}
\subimport{../../contrib/translate/en/manual/contrib-how-to/}{src-code.tex}

\section{History}

\small

\begin{tdocbreak}[version = 1.1.0, date = 2025-10-14]
    \item Duplicate palettes and those that are reverse of others are ignored (strict equalities only).
\end{tdocbreak}


\begin{tdocnew}
    \item New palettes added: \verb+BurningGrass+, \verb+GeoRainbow+ and \verb+PastelRainbow+ (\tdocpack{luadraw} package creation process used).

    \item The \tdocpack{luadraw} palette product has a new dictionary like variable \tdoccodein{lua}{getPal} to access a palette using its name (as a string variable).
\end{tdocnew}


\begin{tdocupdate}
    \item Palette contributions: in the mandatory \verb+extend.py+ file, \tdoccodein{py}{build_code} must work with the dictionary of all the palettes, and manage a credit to the \thisproj\ project.
\end{tdocupdate}

\tdocsep

\tdocstartproj{First public version of the project.}\tdocversion{1.0.0}[2025-10-11]


\end{document}
