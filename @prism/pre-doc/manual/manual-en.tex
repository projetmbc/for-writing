% !TEX TS-program = lualatex

\documentclass{tutodoc}

\usepackage{../../contrib/translate/en/manual/preamble.cfg}

\usepackage[subpreambles = true]{standalone}

\makeatletter
\renewcommand{\minted@cachedir}{_minted-xtra-cache}
\makeatother

\begin{document}

\title{\thisproj*\ project}
\author{Christophe BAL}
\date{9 Nov 2025 -- Version 1.2.1}

\maketitle

\begin{abstract}
    The \thisproj\ project\,%
    \footnote{
        The name comes from \tdocquote{%
        \tdocprewhy{@.esthetic} \tdocprewhy{P.roducts} 
        for
        \tdocprewhy{R.epresenting} \tdocprewhy{I.nformative}
        \tdocprewhy{S.cientific} \tdocprewhy{M.aps}}.
        %
        This name is a double play on words:\kern3pt%
        \begin{enumerate*}[label={[\arabic*]}]
            \item a prism splits light into an informative spectrum, symbolizing how data are decomposed into meaningful color,
            and
            \item "@" read as "at" indicates where the light meets the prism to be broken down into an informative spectrum.
        \end{enumerate*}
    }
    provides small size color palettes that can be used to create expressive color maps for graphics in different contexts.

%    \smallskip
%
%    \noindent
%    \emph{\textbf{Remark :} this documentation is also available in <<DOC-LANGS>>.}
\end{abstract}

\tdocsep

{
\small

\bgroup
    \addtokomafont{subsection}{\centering}
    \subsection*{Last changes}
\egroup

\begin{tdocfix}[version = 1.2.1, date = 2025-11-09]
    \item Equal palettes: the floating point equality uses now a correct tolerance.
\end{tdocfix}


\begin{tdocbreak}
    \item Palettes: the extra \tdoccodein{text}+Greys+ has been removed (it is the reverse of \tdoccodein{text}+Binary+).
\end{tdocbreak}


\begin{tdocnew}
    \item Similar palettes: two PDF files show similar palettes in standard and black modes (semi-automated process used).
\end{tdocnew}


\begin{tdocupdate}
    \item \LUADRAW\ product: the associative array \tdoccodein{lua}+palNames+ has been added for compatibility reasons with the \LUADRAW\ package.


    \item \tdoccodein{text}+BlindFish+ palette: the last color variation has been made smoother (\LUADRAW\ process used).
\end{tdocupdate}
}

\newpage
\tableofcontents
\newpage

\subimport{../../contrib/translate/en/manual/to-know/}{motivations.tex}
\subimport{../../contrib/translate/en/manual/to-know/}{backstage.tex}
\subimport{../../contrib/translate/en/manual/to-know/}{reuse-from.tex}
\subimport{../../contrib/translate/en/manual/to-know/}{searching.tex}
\subimport{../../contrib/translate/en/manual/products/}{preamble.tex}
\subimport{../../contrib/translate/en/manual/products/}{luadraw.tex}
\subimport{../../contrib/translate/en/manual/contrib-how-to/}{prologue.tex}
\subimport{../../contrib/translate/en/manual/contrib-how-to/}{translate.tex}
\subimport{../../contrib/translate/en/manual/contrib-how-to/}{src-code.tex}

\section{History}

\small

\begin{tdocfix}[version = 1.2.1, date = 2025-11-09]
    \item Equal palettes: the floating point equality uses now a correct tolerance.
\end{tdocfix}


\begin{tdocbreak}
    \item Palettes: the extra \tdoccodein{text}+Greys+ has been removed (it is the reverse of \tdoccodein{text}+Binary+).
\end{tdocbreak}


\begin{tdocnew}
    \item Similar palettes: two PDF files show similar palettes in standard and black modes (semi-automated process used).
\end{tdocnew}


\begin{tdocupdate}
    \item \LUADRAW\ product: the associative array \tdoccodein{lua}+palNames+ has been added for compatibility reasons with the \LUADRAW\ package.


    \item \tdoccodein{text}+BlindFish+ palette: the last color variation has been made smoother (\LUADRAW\ process used).
\end{tdocupdate}

\tdocsep

\begin{tdocbreak}[version = 1.2.0, date = 2025-10-29]
    \item Palettes: all final palettes now consist of 10 colors.


    \item \LUADRAW\ products: the \tdoccodein{lua}+getPal+ dictionary array has been converted into a function accepting string palette names (with or without \tdoccodein{lua}+pal+ prefix). See below.
\end{tdocbreak}


\begin{tdocnew}
    \item Palettes.
    %
    \begin{itemize}
        \item Added  \tdoccodein{text}+Lemon+ and \tdoccodein{text}+ShiftRainbow+ palettes (\LUADRAW\ creation process used).

        \item Added 37 palettes from the \scicolmap\ project.
    \end{itemize}


    \item \LUADRAW\ product: accessing a palette and creating new ones can be made using the \tdoccodein{lua}+getPal+ function which has an optional argument \tdoccodein{lua}+options+ (dict-like array) with the following keys and their values.
    %
    \begin{itemize}
        \item \tdoccodein{lua}+extract+: a list of non-zero integers used to extract specific colors from the palette (the order is preserved).

        \item \tdoccodein{lua}+reverse+: a boolean value indicating whether to reverse the palette color order (\tdoccodein{lua}+false+ by default).

        \item \tdoccodein{lua}+shift+: an integer value for applying a circular color shift to the palette.
    \end{itemize}
    

    \item Documentations
    %
    \begin{itemize}
        \item Added English PDF manual.

        \item Showcase: two PDF files demonstrate the use of each palette (white and dark modes).
    \end{itemize}
\end{tdocnew}

\tdocsep

\begin{tdocbreak}[version = 1.1.0, date = 2025-10-14]
    \item Duplicate palettes and those that are reverse of others are ignored (strict equalities only).
\end{tdocbreak}


\begin{tdocnew}
    \item New palettes added: \tdoccodein{text}+BurningGrass+, \tdoccodein{text}+GeoRainbow+ and \tdoccodein{text}+PastelRainbow+ (\LUADRAW\ creation process used).

    \item The \LUADRAW\ palette product has a new dictionary like variable \tdoccodein{lua}{getPal} to access a palette using its name (as a string variable).
\end{tdocnew}


\begin{tdocupdate}
    \item Palette contributions: in the mandatory \verb+extend.py+ file, the \tdoccodein{py}{build_code} function must work with the dictionary of all the palettes, and manage a credit to the \thisproj\ project.
\end{tdocupdate}

\tdocsep

\tdocstartproj{First public version of the project.}\tdocversion{1.0.0}[2025-10-11]


\end{document}
