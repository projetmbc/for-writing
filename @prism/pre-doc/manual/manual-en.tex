% !TEX TS-program = lualatex

\documentclass{tutodoc}

\usepackage{../../contrib/translate/en/manual/preamble.cfg}

\usepackage[subpreambles = true]{standalone}

\makeatletter
\renewcommand{\minted@cachedir}{_minted-xtra-cache}
\makeatother

\begin{document}

\title{\thisproj{} project}
\author{Christophe BAL}
\date{29 Oct 2025 -- Version 1.2.0}

\maketitle

\begin{abstract}
    The \thisproj{} project\,%
    \footnote{
        The name comes from \tdocquote{%
        \tdocprewhy{@.esthetic} \tdocprewhy{P.roducts} 
        for
        \tdocprewhy{R.epresenting} \tdocprewhy{I.nformative}
        \tdocprewhy{S.cientific} \tdocprewhy{M.aps}}.
        %
        This name is a double play on words:\kern3pt%
        \begin{enumerate*}[label={[\arabic*]}]
            \item a prism is where light is split into an informative spectrum, symbolizing how data or visuals are decomposed into meaningful color and style,
            and
            \item where light meets the prism, it breaks down into an informative spectrum ("@" can be read "at").
        \end{enumerate*}
    }
    provides small size color palettes that can be used to create expressive color maps for graphics in different contexts.

%    \smallskip
%
%    \noindent
%    \emph{\textbf{Remark :} this documentation is also available in <<DOC-LANGS>>.}
\end{abstract}

\tdocsep

{
\small

\bgroup
    \addtokomafont{subsection}{\centering}
    \subsection*{Last changes}
\egroup

\begin{tdocbreak}[version = 1.2.0, date = 2025-10-29]
    \item Palettes: all final palettes now consist of 10 colors.


    \item \LUADRAW\ products: the \tdoccodein{lua}+getPal+ dictionary array has been converted into a function accepting string palette names (with or without \tdoccodein{lua}+pal+ prefix). See below.
\end{tdocbreak}


\begin{tdocnew}
    \item Palettes.
    %
    \begin{itemize}
        \item Added  \tdoccodein{text}+Lemon+ and \tdoccodein{text}+ShiftRainbow+ palettes (\LUADRAW\ creation process used).

        \item Added 37 palettes from the \scicolmap\ project.
    \end{itemize}


    \item \LUADRAW\ product: the \tdoccodein{lua}+getPal+ function has an optional argument \tdoccodein{lua}+options+ (dict-like array).
    %
    \begin{itemize}
        \item \tdoccodein{lua}+extract+ key allows to give a list of non-zero integers to extract specific colors (preserves order).

        \item \tdoccodein{lua}+reverse+ key allows to give a boolean to reverse palette color order, or not (\tdoccodein{lua}+false+ by defaul)

        \item \tdoccodein{lua}+shift+ key allows to give an integer for circular color shift.
    \end{itemize}


    \item Documentations
    %
    \begin{itemize}
        \item Added English PDF manual.

        \item Showcase: two PDF files demonstrate the use of each palette (white and dark modes).
    \end{itemize}
\end{tdocnew}
}

\newpage
\tableofcontents
\newpage

\subimport{../../contrib/translate/en/manual/to-know/}{motivations.tex}
\subimport{../../contrib/translate/en/manual/to-know/}{backstage.tex}
\subimport{../../contrib/translate/en/manual/to-know/}{reuse-from.tex}
\subimport{../../contrib/translate/en/manual/to-know/}{searching.tex}
\subimport{../../contrib/translate/en/manual/products/}{preamble.tex}
\subimport{../../contrib/translate/en/manual/products/}{luadraw.tex}
\subimport{../../contrib/translate/en/manual/contrib-how-to/}{prologue.tex}
\subimport{../../contrib/translate/en/manual/contrib-how-to/}{translate.tex}
\subimport{../../contrib/translate/en/manual/contrib-how-to/}{src-code.tex}

\section{History}

\small

\begin{tdocbreak}[version = 1.2.0, date = 2025-10-29]
    \item Palettes: all final palettes now consist of 10 colors.


    \item \LUADRAW\ products: the \tdoccodein{lua}+getPal+ dictionary array has been converted into a function accepting string palette names (with or without \tdoccodein{lua}+pal+ prefix). See below.
\end{tdocbreak}


\begin{tdocnew}
    \item Palettes.
    %
    \begin{itemize}
        \item Added  \tdoccodein{text}+Lemon+ and \tdoccodein{text}+ShiftRainbow+ palettes (\LUADRAW\ creation process used).

        \item Added 37 palettes from the \scicolmap\ project.
    \end{itemize}


    \item \LUADRAW\ product: the \tdoccodein{lua}+getPal+ function has an optional argument \tdoccodein{lua}+options+ (dict-like array).
    %
    \begin{itemize}
        \item \tdoccodein{lua}+extract+ key allows to give a list of non-zero integers to extract specific colors (preserves order).

        \item \tdoccodein{lua}+reverse+ key allows to give a boolean to reverse palette color order, or not (\tdoccodein{lua}+false+ by defaul)

        \item \tdoccodein{lua}+shift+ key allows to give an integer for circular color shift.
    \end{itemize}


    \item Documentations
    %
    \begin{itemize}
        \item Added English PDF manual.

        \item Showcase: two PDF files demonstrate the use of each palette (white and dark modes).
    \end{itemize}
\end{tdocnew}

\tdocsep

\begin{tdocbreak}[version = 1.1.0, date = 2025-10-14]
    \item Duplicate palettes and those that are reverse of others are ignored (strict equalities only).
\end{tdocbreak}


\begin{tdocnew}
    \item New palettes added: \verb+BurningGrass+, \verb+GeoRainbow+ and \verb+PastelRainbow+ (\tdocpack{luadraw} package creation process used).

    \item The \tdocpack{luadraw} palette product has a new dictionary like variable \tdoccodein{lua}{getPal} to access a palette using its name (as a string variable).
\end{tdocnew}


\begin{tdocupdate}
    \item Palette contributions: in the mandatory \verb+extend.py+ file, \tdoccodein{py}{build_code} must work with the dictionary of all the palettes, and manage a credit to the \thisproj\ project.
\end{tdocupdate}

\tdocsep

\tdocstartproj{First public version of the project.}\tdocversion{1.0.0}[2025-10-11]


\end{document}
