% !TEX TS-program = lualatex

\documentclass{tutodoc}

\usepackage{../../preamble.cfg}


\begin{document}

\small


\begin{tdocbreak}
    \item Palettes: all final palettes now consist of 10 colors.


    \item \LUADRAW\ products: the \tdoccodein{lua}+getPal+ dictionary array has been converted into a function accepting string palette names (with or without \tdoccodein{lua}+pal+ prefix). See below.
\end{tdocbreak}


\begin{tdocnew}
    \item Palettes.
    %
    \begin{itemize}
        \item Added  \tdoccodein{text}+Lemon+ and \tdoccodein{text}+ShiftRainbow+ palettes (\LUADRAW\ creation process used).

        \item Added 37 palettes from the \scicolmap\ project.
    \end{itemize}


    \item \LUADRAW\ product: the \tdoccodein{lua}+getPal+ function has an optional argument \tdoccodein{lua}+options+ (dict-like array).
    %
    \begin{itemize}
        \item \tdoccodein{lua}+extract+ key allows to give a list of non-zero integers to extract specific colors (preserves order).

        \item \tdoccodein{lua}+reverse+ key allows to give a boolean to reverse palette color order, or not (\tdoccodein{lua}+false+ by defaul)

        \item \tdoccodein{lua}+shift+ key allows to give an integer for circular color shift.
    \end{itemize}


    \item Documentations
    %
    \begin{itemize}
        \item Added English PDF manual.

        \item Showcase: two PDF files demonstrate the use of each palette (white and dark modes).
    \end{itemize}
\end{tdocnew}

\end{document}
