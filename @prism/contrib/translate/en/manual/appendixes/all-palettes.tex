% !TEX TS-program = lualatex

\documentclass{tutodoc}

\usepackage{../preamble.cfg}

% DEBUG - START
%\geometry{showframe=true}
% DEBUG - END

\begin{document}

\section*{Appendix 1 -- The \palNb\ palettes at a glance} \label{all-palettes}
\addcontentsline{toc}{section}{Appendix 1 -- The \palNb\ palettes at a glance}

The palette names used in this appendix are standard, but most \thisproj\ implementations add the \verb+pal+ prefix.


\begin{tdocimp}
    Categories were generated semi-automatically using a program, followed by manual selection to obtain relevant choices.
    If you identify any errors, please contact the author of \thisproj.
\end{tdocimp}


% ----------------------------------------- %
% -- JUST TRANSLATE TITLES INSIDE BRACES -- %
% ----------------------------------------- %

\foreach \nb/\title in {%
% AUTO CATEGOS - START
  1/{Colorblind-friendly palettes -- 40 palettes},%
  2/{Two-color palettes -- 52 palettes},%
  3/{Three-color palettes -- 54 palettes},%
  4/{Rainbow-style palettes -- 118 palettes},%
  5/{High-contrast palettes -- 34 palettes}%
% AUTO CATEGOS - END
}{
  \ifthenelse{\nb = 1}{}{\newpage}

  \subsection*{\title}
  \addcontentsline{toc}{subsection}{\texorpdfstring{\textbullet\,\title}{\title}}
  %
  \subimport{../../../common/category/}{catego-\nb.luadraw}
}

% DEBUG - START
\end{document}

\parbox[c]{0.3\linewidth}{\directlua{drawSpectrum(getPal("PerceptualRainbow"), "XXX")}}
\hfill
\parbox[c]{3.25cm}{\centering\textbf{\texttt{PerceptualRainbow}} \\ {\footnotesize\color{Grey}Scientific Colour Maps}}
\hfill
\parbox[c]{0.45\linewidth}{\directlua{drawPalette(getPal("PerceptualRainbow"), "XXX")}}



OK?
    
\end{document}
% DEBUG - END
