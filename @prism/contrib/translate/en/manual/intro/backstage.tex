% !TEX TS-program = lualatex

\documentclass{tutodoc}

\usepackage{../preamble.cfg}


\begin{document}

\section{Where do the color palletes come from?}

The vast majority of color palettes were obtained from \matplotlib\ color maps by segmenting them into $10$ values.%
\footnote{
    \asymptote\ is used in a second step, but to date, \asymptote\ offers nothing more than \matplotlib, despite different implementations.
}
We did not keep all the palletes in accordance with the following rules.
%
\begin{itemize}
    \item \matplotlib\ offers a reversed version for almost every color map.%
    \footnote{
        Perhaps for performance reasons...
    }
    \thisproj\ does not make this choice. Specifically, we ignore all color maps ending with \tdoccodein{py}{_r}, this suffix indicating the extra palettes.%
    \footnote{
        At least from the point of view of the \thisproj\ author. 
    }

    \item Some palettes are repeated. In this case, we keep the first one in lexicographical order.%
    \footnote{
        Surely for historical reasons.
    }
\end{itemize}


In addition to the \matplotlib\ palettes, there are some additional contributions. If you are interested in participating, read the \verb+contrib/products/README.md+ file on the following repository.
%
\begin{center}
    \thisrepo
\end{center}

\end{document}
