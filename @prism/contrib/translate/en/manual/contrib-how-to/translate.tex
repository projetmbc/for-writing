% !TEX TS-program = lualatex

\documentclass{tutodoc}

\usepackage{../preamble.cfg}


\NewDocumentCommand{\mailsubject}{m}%  <-- Translate me!
  {subject \tdocquote{\texttt{\thisproj\ - CONTRIB - #1}}}

% Source: https://tex.stackexchange.com/a/424061/6880

\newcommand{\FTdirO}{}
\def\FTdirO(#1,#2,#3){%
  \FTfile(#1,{\color{blue!40!black}\faFolderOpen\hspace{-.35pt}}{\hspace{0.2em}#3})
  (tmp.west)++(0.8em,-0.4em)node(#2){}
  (tmp.west)++(1.5em,0)
  ++(0,-1.3em)
}

\newcommand{\FTdirC}{}
\def\FTdirC(#1,#2,#3){%
  \FTfile(#1,{\color{blue!40!black}\faFolder\hspace{.75pt}}{\hspace{0.2em}#3})
  (tmp.west)++(0.8em,-0.4em)node(#2){}
  (tmp.west)++(1.5em,0)
  ++(0,-1.3em)
}

\newcommand{\FTfile}{}
\def\FTfile(#1,#2){%
  node(tmp){}
  (#1|-tmp)++(0.6em,0)
  node(tmp)[anchor=west,black]{\texttt{#2}}
  (#1)|-(tmp.west)
  ++(0,-1.2em)
}

\newcommand{\FTroot}{}
\def\FTroot{tmp.west}

\newcommand\contribtranslatedirtree{
  \begin{tikzpicture}%
    \draw[color=black, thick]
% en        : parent = \FTroot
% normal dir: (parentID, currentID, label)
% file      :       (parentID, label)
      \FTdirO(\FTroot,root,translate){
        \FTdirC(root,changes,changes){
        }
        \FTdirO(root,en,en) {
          \FTdirC(en,manual,manual)
        }
        \FTdirC(root,status,status){
          \FTdirO(status,en,en) {
            \FTfile(en,manual.yaml)
          }
        }
        \FTfile(root,README.md)
        \FTfile(root,LICENCE.txt)
      };
  \end{tikzpicture}
}


\begin{document}

%\section{Contribute}

\subsection{Complete the translations}

\begin{tdocimp}
    Although we're going to explain how to translate the documentation, it doesn't seem relevant to do so, as English should suffice these days.
\end{tdocimp}


\begin{figure}[ht]
    \centering
    \contribtranslatedirtree\
    \caption{Simplified view of the translation folder}
    \label{tutodoc-contrib-translate-dir}
\end{figure}


The translations are roughly organized as in figure \ref{tutodoc-contrib-translate-dir} where just the important folders for the translations have been \tdocquote{opened}\,.%
\footnote{
    This was the organization on October 26, 2025.
}
\textbf{A little further down, the section \ref{tutodoc-contrib-translate} explains how to add new translations}.


\subsubsection{The \texttt{en} folder}

This folder, managed by the author of \thisproj, contains files easy to translate even if you're not a coder.


\subsubsection{The \texttt{changes} folder}

This folder is a communication tool where important changes are indicated without dwelling on minor modifications specific to one or more translations.


\subsubsection{The \texttt{status} folder}

This folder is used to keep track of translations from the project's point of view. Everything is done via well-commented \verb#YAML# files, readable by a non-coder.


\subsubsection{The \texttt{README.md} and \texttt{LICENCE.txt} files}

The \texttt{LICENCE.txt} file is aptly named, while the \texttt{README.md} file takes up in English the important points of what is said in this section about new translations.


\subsubsection{New translations}
\label{tutodoc-contrib-translate}

\begin{tdocnote}
    The folder \verb#manual# is reserved for documentation. It contains \verb#TEX# files that can be compiled directly for real-time validation of translations.
\end{tdocnote}


\begin{tdocwarn}
    Only start from the \verb#en# folder, as it's the responsibility of the \thisproj\ author.
\end{tdocwarn}


\medskip


\emph{\textbf{Let's say you want to add support for Italian from files written in English.}}%
\footnote{
    As mentioned above, there is no real need for the \texttt{doc} folder.
}
To do this, you must use \git\ as follows.
%
\begin{enumerate}
    \item Via \thisrepo, recover the entire project folder.
    Do not use the \verb#main# branch, which is used to freeze the latest stable versions of projects in the single \thismonorepo\ repository,.

    \item In the \verb#@prism/contrib/translate# folder, create an \verb#it# copy of the \verb#en# folder, with the short name of the language documented in
    \href{https://en.wikipedia.org/wiki/IETF_language_tag#List_of_common_primary_language_subtags}%
         {the page \tdocquote{IIETF language tag}}
    from \texttt{Wikipedia}.

    \item Once the translation is complete in the \verb#it# folder, share it via \thisrepo\ using a classic \verb#git push#\,.
\end{enumerate}

\end{document}
