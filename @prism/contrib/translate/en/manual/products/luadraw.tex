% !TEX TS-program = lualatex

\documentclass{tutodoc}

\usepackage{../preamble.cfg}


\begin{document}

\subsection{luadraw palettes}
\label{products-luadraw}

\subsubsection{Description}

You can use {\thisproj} palettes with {\LUADRAW} which is a package that greatly facilitates the creation of high-quality 2D and 3D plots via {\lua} and {\TikZ}.

\begin{tdocnote}
Initially, the {\thisproj} project was created to provide ready-to-use palettes for {\LUADRAW}.
\end{tdocnote}

\subsubsection{Use a luadraw palette}

The {\lua} palette names all use the prefix \tdoccodein{text}+pal+ followed by the name available in the file \tdoccodein{text}+palettes.json+. You can access a palette by two ways.

\begin{itemize}
\item
\tdoccodein{lua}+palGistHeat+ is a {\lua} variable.

\item
\tdoccodein{lua}+getPal('GistHeat')+ and \tdoccodein{lua}+getPal('palGistHeat')+ are equal to \tdoccodein{lua}+palGistHeat+.
\end{itemize}

\begin{tdocnote}
The {\lua} palette variables are arrays of arrays of three floats. Here is the definition of \tdoccodein{lua}+palGistHeat+.

\begin{tdoccode}{lua}
palGistHeat = {
    {0.0, 0.0, 0.0},
    {0.105882, 0.0, 0.0},
    {0.211764, 0.0, 0.0},
    {0.317647, 0.0, 0.0},
    {0.429411, 0.0, 0.0},
    {0.535294, 0.0, 0.0},
    {0.641176, 0.0, 0.0},
    {0.752941, 0.003921, 0.0},
    {0.858823, 0.145098, 0.0},
    {0.964705, 0.286274, 0.0},
    {1.0, 0.42745, 0.0},
    {1.0, 0.57647, 0.152941},
    {1.0, 0.717647, 0.435294},
    {1.0, 0.858823, 0.717647},
    {1.0, 1.0, 1.0}
}
\end{tdoccode}
\end{tdocnote}

There are also some options. To explain how this works, let's consider the following use case.

\begin{tdoccode}{lua}
mypal = getPal(
    'GistHeat',
    {
        extract = {2, 5, 8, 9},
        shift   = 1,
        reverse = true
    }
)
\end{tdoccode}

To simplify the explanations, we will refer to the colors in the standard palette \tdoccodein{text}+'GistHeat'+ as \tdoccodein{lua}+coul_1+, \tdoccodein{text}+coul_2,+, etc. The options are then processed in the following order.

\begin{enumerate}
\item
\tdoccodein{text}+{coul_2, coul_5, coul_8, coul_9}+ is the result of the extraction.

\item
\tdoccodein{text}+{coul_9, coul_2, coul_5, coul_8}+ comes from the shifting applied to the extracted palette (colors move to the right if \tdoccodein{text}+shift+ is positive).

\item
\tdoccodein{text}+{coul_8, coul_5, coul_2, coul_9}+ is the reversed version of the shifted palette.

\end{enumerate}

\end{document}
