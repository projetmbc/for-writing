% !TEX TS-program = lualatex

% ------------------------------------------------- %
% -- DO NOT MODIFY. FILE GENERATED AUTOMATICALLY -- %
% ------------------------------------------------- %

\documentclass{tutodoc}

\usepackage{../preamble.cfg}


\begin{document}

% -- AUTOMATICALLY GENERATED UGLY CODE - START -- %

\subsection{lua palettes}
\label{products-lua}

\begin{tdocnote}
Initially, the {\thisproj} project was created to provide ready-to-use palettes for {\LUADRAW} which is a package that greatly facilitates the creation of high-quality 2D and 3D plots via {\LuaLaTeX} and {\TIKZ}.
\end{tdocnote}

\subsubsection{Use a lua palette}

The {\lua} palette names all use the prefix \tdoccodein{text}+pal+ followed by the name available in the file \tdoccodein{text}+palettes.json+. You can access a palette by three ways.

\begin{itemize}
\item
\tdoccodein{lua}+palGistHeat+ is a {\lua} variable.

\item
\tdoccodein{lua}+getPal('GistHeat')+ and \tdoccodein{lua}+getPal('palGistHeat')+ are equal to \tdoccodein{lua}+palGistHeat+.

\item
\tdoccodein{lua}+palNames['palGistHeat']+ is equal to \tdoccodein{lua}+palGistHeat+.
\end{itemize}

\begin{tdocnote}
The {\lua} palette variables are arrays of arrays of three floats. The definition of \tdoccodein{lua}+palGistHeat+ looks like the following partial code.

\begin{tdoccode}{lua}
palGistHeat = {
    {0.0, 0.0, 0.0},
    {0.105882, 0.0, 0.0},
    {0.211764, 0.0, 0.0},
    -- ... With 7 more RBG colors.
}
\end{tdoccode}
\end{tdocnote}

The \tdoccodein{lua}+getPal+ function has some options. To explain how this works, let's consider the following use case.

\begin{tdoccode}{lua}
mypal = getPal(
    'GistHeat',
    {
        extract = {2, 5, 8, 9},
        shift   = 1,
        reverse = true
    }
)
\end{tdoccode}

To simplify the explanations, we will refer to the colors in the standard palette \tdoccodein{text}+'GistHeat'+ as \tdoccodein{lua}+coul_1+, \tdoccodein{lua}+coul_2+, etc. The options are then \textbf{processed in the following order}.

\begin{enumerate}
\item
\tdoccodein{text}+{coul_2, coul_5, coul_8, coul_9}+ is the result of the extraction.

\item
\tdoccodein{text}+{coul_9, coul_2, coul_5, coul_8}+ comes from the shifting applied to the extracted palette (colors move to the right if \tdoccodein{text}+shift+ is positive).

\item
\tdoccodein{text}+{coul_8, coul_5, coul_2, coul_9}+ is the reversed version of the shifted palette.

\end{enumerate}

\begin{tdocnote}
The reversed version of any palette can be obtained using \tdoccodein{lua}+getPal(palname, {reverse = true})+.
\end{tdocnote}

% -- AUTOMATICALLY GENERATED UGLY CODE - END -- %

\end{document}
