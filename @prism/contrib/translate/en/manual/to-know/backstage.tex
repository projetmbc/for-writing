% !TEX TS-program = lualatex

\documentclass{tutodoc}

\usepackage{../preamble.cfg}


\begin{document}

\section{Where do the color palettes come from?}

Most color palettes are obtained from \matplotlib\ and \scicolmap\ by segmenting their color maps into $\palSize$ values.%
\footnote{
    \asymptote\ is also used, but currently offers nothing beyond \matplotlib, despite different implementations.
}
We retain only palettes that comply with the following rules.
%
\begin{itemize}
    \item \textbf{No repetition.}
    Some \matplotlib\ palettes are duplicated,%
    \footnote{
        Likely for historical reasons.
    }
    in which case we keep the first one in lexicographical order.
    
    \item \textbf{No reversed versions.}
    Unlike \matplotlib,%
    \footnote{
    	Most \matplotlib\ color maps have a reversed version named with the \tdoccodein{py}{_r} suffix, possibly for performance reasons.
    }
    \thisproj\ never includes reversed palettes as fixed data.
\end{itemize}


\begin{tdocimp}
    \matplotlib\ integrates all palettes from \colorbrewer.
\end{tdocimp}


% -------------------- %


In addition to \matplotlib\ and \scicolmap\ palettes, \thisproj\ includes some original creations.

\begin{tdocnote}
    Adding new palettes to \thisproj\ is straightforward (no coding skills required).
    See section \ref{contrib-how-to-src-code} to get started.
\end{tdocnote}



% -------------------- %


We list below the palettes ignored due to duplication.%
\footnote{%
    Recall that \matplotlib\ reversed color maps (with the \tdoccodein{py}{_r} suffix) are systematically excluded and therefore not shown here.
}
The symbol \boxed{{=}\vphantom{pM}} indicates equality, \boxed{{\rightleftharpoons}\vphantom{pM}} indicates reversal, and the rightmost palette is the one retained in \thisproj.
%
\subimport{../../../common/}{ignored-palettes.latex}

\end{document}
