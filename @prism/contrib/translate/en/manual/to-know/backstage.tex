% !TEX TS-program = lualatex

\documentclass{tutodoc}

\usepackage{../preamble.cfg}


\begin{document}

\section{Where do the color palettes come from?}

Most of color palettes are obtained from \matplotlib\ and \scicolmap\ by segmenting their color maps into $15$ values.%
\footnote{
    \asymptote\ is also used, but to date, \asymptote\ offers nothing more than \matplotlib, despite different implementations.
}
We don't keep all the palettes in accordance with the following rules.
%
\begin{itemize}
    \item \textbf{No repetition.}
    Some \matplotlib\ palettes are repeated.%
    \footnote{
        Surely for historical reasons.
    }
    In this case, we keep the first one regarding to the lexicographical order.


    \item \textbf{No reversed version.}
    Contrary to \matplotlib,%
    \footnote{
    	Most of the \matplotlib\ color maps have a reversed version named by adding the suffix \tdoccodein{py}{_r}.
	    Perhaps this is for performance reasons...
    }
    \thisproj\ never offers the reversed version of a palette as a fixed data.
\end{itemize}


In addition to the \matplotlib\ and \scicolmap\ palettes, there are \thisproj\ creations.


\begin{tdocnote}
    If you are interested in adding palettes, go to the section \ref{contrib-how-to-src-code}.
\end{tdocnote}

\end{document}
