% !TEX TS-program = lualatex

\documentclass{tutodoc}

\usepackage{../preamble.cfg}


\begin{document}

\title{\thisproj{} project}
\author{<<AUTHOR>>}
\date{<<DATE-N-VERSION>>}

\maketitle

\begin{abstract}
    The \thisproj{} project\,%
    \footnote{
        The name comes from \tdocquote{%
        \tdocprewhy{@.esthetic} \tdocprewhy{P.roducts} 
        for
        \tdocprewhy{R.epresenting} \tdocprewhy{I.nformative}
        \tdocprewhy{S.cientific} \tdocprewhy{M.aps}}.
        %
        This name is a double play on words:\kern3pt%
        \begin{enumerate*}[label={[\arabic*]}]
            \item a prism is where light is split into an informative spectrum, symbolizing how data or visuals are decomposed into meaningful color and style,
            and
            \item where light meets the prism, it breaks down into an informative spectrum ("@" can be read "at").
        \end{enumerate*}
    }
    provides small size color palettes that can be used to create expressive color maps for graphics in different contexts.

%    \smallskip
%
%    \noindent
%    \emph{\textbf{Remark :} this documentation is also available in <<DOC-LANGS>>.}
\end{abstract}

\end{document}
