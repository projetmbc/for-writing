% !TEX TS-program = lualatex

\documentclass{tutodoc}

\usepackage{../preamble.cfg}


\begin{document}

\section{Where do the color palettes come from?}
\label{projects-used}

\thisproj\ includes some original creations, but most color palettes are derived from the projects listed below by segmenting their color maps into $\palSize$-value palettes.
%
% ---------------------------------------- %
% -- JUST TRANSLATE THE ''XTRA'' TEXTS. -- %
% ---------------------------------------- %
\def\frompalettable{palettes are extracted from \palettable\ project}
%
\begin{multicols}{2}
\begin{itemize}
  \foreach \techno/\xtra in {%
    % -- A -- %
    \asymptote / {is used, but currently offers nothing beyond \matplotlib\ (despite different implementations)},%
    % -- C -- %
    \cartocolor  / \frompalettable,%
    \cmocean     / \frompalettable,%
    \colorbrewer / {provides professional color palettes for mapping and data visualization},%
    % -- L -- %
    \lightbartlein / \frompalettable,%
    % -- M -- %
    \matplotlib / {compiles color maps from diverse projects, serving as the foundation for the initial palette list}, %
    \mycarta    / \frompalettable,%
    % -- P -- %
    \plotly / \frompalettable,%
    % -- S -- %
    \scicolmap / {provides palettes designed for colorblind accessibility}, %
    % -- W -- %
    \tableau / \frompalettable, %
    % -- W -- %
    \wesanderson / \frompalettable %
  }{%
    \item \techno\xtra.
  }
\end{itemize}
\end{multicols}

% -------------------- %


We retain only palettes that comply with the following rules.
%
\begin{itemize}
    \item \textbf{No repetition.}
    Unlike \matplotlib,%
    \footnote{
        Some \matplotlib\ palettes are duplicated, likely for historical reasons.
    }
    \thisproj\ use a one-to-one map from names to palettes.

    \item \textbf{No reversed versions.}
    Unlike \matplotlib,%
    \footnote{
    	Most \matplotlib\ color maps have a reversed version named with the \tdoccodein{py}{_r} suffix, possibly for performance reasons.
    }
    \thisproj\ never includes reversed palettes as fixed data.
\end{itemize}


% -------------------- %


\begin{tdocnote}
    Adding new palettes to \thisproj\ is straightforward (no coding skills required).
    See section \ref{contrib-how-to-src-code} to get started.
\end{tdocnote}


\end{document}
