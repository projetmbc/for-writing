\documentclass{tutodoc}

\usepackage{../preamble.cfg}


\begin{document}

\title{Le projet \thisproj{}}
\author{<<AUTHOR>>}
\date{<<DATE-N-VERSION>>}

\maketitle

\begin{abstract}
    Le projet \thisproj{}\,%
    \footnote{
        Le nom vient de \tdocquote{\tdocprewhy{A.esthetic} \tdocprewhy{T.oolkit} - \tdocprewhy{P.alettes} to \tdocprewhy{R.ender} \tdocprewhy{I.llustrative} \tdocprewhy{S.cientific} \tdocprewhy{M.aps}}, ceci se traduisant en \tdocquote{Boîte à outils esthétique - Palettes pour le rendu
        d'application scientifique illustrative}.
        Ce nom est aussi un jeu de mots: c'est au niveau du prisme que la lumière est décomposée en un spectre.
    }
    XXXX

%    \smallskip
%
%    \noindent
%    \emph{\textbf{Remarque :} cette documentation est aussi disponible en <<DOC-LANGS>>.}
\end{abstract}

\end{document}
