% !TEX TS-program = lualatex

%%%
% Test your palette here. Once you are satisfied with your work,
% copy and paste the file ''PROJECT-PALETTE.lua'' into the
% ''contrib/palettes/luadraw/palettes'' folder, giving it a name
% in ''UpperCamelCase” format. This file only uses floats such
% as to obtain portable palette definitions.
%
%
% caution::
%     You can use any luadraw colors, but you can't change the
%     variable name ''PALETTE'' needed to automate some tasks.
%
%
% note::
%     In the Lua palette file, the ''author'' field is optional.
%%%

\begin{filecontents*}[overwrite]{__tmp-palette__.lua}
-- author: First Name, Last Name

PALETTE = {
  DarkGreen,
  Green,
  Yellow,
  Orange,
  Red,
}
\end{filecontents*}


% ----------------------- %
% -- PALETTE RENDERING -- %
% ----------------------- %

\documentclass{scrartcl}

\usepackage[
  landscape,
  twocolumn,
  top  = .75cm, bottom = .75cm,
  left = .75cm, right  = .75cm
]{geometry}

\usepackage[svgnames]{xcolor}
\usepackage[3d]{luadraw}

\usepackage[most]{tcolorbox}
\tcbuselibrary{minted, skins, breakable}

\newtcbinputlisting{\showlua}[1]{%
% Listing options.
  listing only,
  listing file    = {#1},
  listing engine  = minted,
  minted language = lua,
  minted options  = {
%    linenos,
    breaklines,
    autogobble,
  },
% Colors.
  colback  = yellow!5!white,
  colframe = orange!50!black,
% Frame.
  arc     = 2pt,
  boxrule = 0.8pt,
  left    = 1mm,
  right   = 1mm,
  top     = 1mm,
  bottom  = 1mm,
% Misc.
%  enhanced,
%  breakable,
}

\begin{document}

% -------------------------- %
% -- COMMON LUA CODE USED -- %
% -------------------------- %

\directlua{dofile('core/drawsurf-z-bary.lua')}
\directlua{dofile("__tmp-palette__.lua")}


% ------------------- %
% -- STANDARD CODE -- %
% ------------------- %

\begin{luacode*}
local nb_colors = #PALETTE

local file = io.open(
  "PROJECT-PALETTE.lua",
  "w"
)

file:write("PALETTE = {\n")

for n , rgb in ipairs(PALETTE) do
  file:write("  {")

  for i, c in ipairs(rgb) do
    file:write(c)

    if i ~= 3 then
      file:write(", ")
    end
  end

  file:write("}")

  if n ~= nb_colors then
    file:write(",")
  end

  file:write("\n")
end

file:write("}\n")

file:close()
\end{luacode*}


% -------------- %
% -- SHOWCASE -- %
% -------------- %

\section*{Your palette definition}

\showlua{__tmp-palette__.lua}


\section*{Use cases}

\centering

\begin{luadraw}{name = palette-TEST}
local xmin, xmax = -5, 5

local graphview = graph:new{
  window = {xmin, xmax, -1, 1},
  size   = {10, 10},
  margin = {0.05, 0.05, 0.05, 0.05},
  border = true
}

local i = cpx.I
local N = 200

graphview:Linewidth(18)

local pos, x

for k = 1, N do
  pos = (k - 1) / (N - 1)
  x   = xmin + 10*pos

  graphview:Dpolyline(
    {x - i, x + i},
    "color = " .. palette(PALETTE, pos)
  )
end

graphview:Show()
\end{luadraw}


\bigskip


\begin{luadraw}{name = contour-TEST}
local graphview = graph:new{
  window = {-1, 6.5, -1.5, 11},
  size   = {8, 8, 2},
  margin = {.25, 0, 0, 0},
  scale  = 1.5,
  border = true
}

local i, sin, cos = cpx.I, math.sin, math.cos

local f = function(x, y)
  return (x + y) / (2 + cos(x)*sin(y))
end

local Lz = range(1, 10)

local Colors = getpalette(PALETTE, 10)

graphview:Dgradbox(
  {0, 5 + 10*i, 1, 1},
  {
    legend = {"$x$", "$y$"},
    grid   = true,
    title  = "$z = \\frac{x + y}{2 + \\cos(x) \\sin(y)}$"
  }
)

graphview:Linewidth(20)

graphview:Dcontour(
  f,
  Lz,
  {
    view   = {0, 5, 0, 10},
    colors = Colors
  }
)

for k = 1, 10 do
  local y = (2*k + 4) / 3*i

  graphview:Dseg(
    {5.25 + y, 5.5 + y},
    1,
    "color = " .. Colors[k]
  )

  graphview:Labelcolor(Colors[k])

  graphview:Dlabel(
    "$z = " .. k .. "$",
    5.5 + y,
    {pos = "E"}
  )
end

graphview:Show()
\end{luadraw}


\bigskip


\begin{luadraw}{name = level-curve-surface-TEST}
local cos, sin = math.cos, math.sin, math.pi

local graphview = graph3d:new{
  window3d = {0, 5, 0, 10, 0, 11},
  adjust2d = true,
  size     = {9, 9},
  viewdir  = {220, 60},
  border   = true
}

graphview:Labelsize("footnotesize")

local S = surface(
  function(u, v)
    return M(u, v, (u + v) / (2 + cos(u)*sin(v)))
  end,
  0, 5, 0, 10,
  {30, 30}
)

local n = 10

local Colors = getpalette(PALETTE, n, true)

local niv, S1 = {}

for k = 1, n do
  S1, S = cutfacet(S, {M(0, 0, k), -vecK})

  insert(
    niv,
    {
      S1,
      {
        color     = Colors[k],
        mode      = mShaded,
        edgewidth = 0.5
      }
    }
  )
end

insert(
  niv,
  {
    S,
    {color = Colors[n + 1]}
  }
)

graphview:Dboxaxes3d({
  grid      = true,
  gridcolor = "gray",
  fillcolor = "lightgray"
})

graphview:Dmixfacet(table.unpack(niv))

for k = 1, n do
  graphview:Dballdots3d(
    M(5, 0, k),
    rgb(Colors[k])
  )
end

graphview:Dlabel(
  "$z = \\frac{x + y}{2 + \\cos(x) \\sin(y)}$",
  Z(
    (graphview:Xinf() + graphview:Xsup())/2 ,
    graphview:Yinf()
  ),
  {pos = "N"}
)

graphview:Show()
\end{luadraw}


\bigskip


\begin{luadraw}{name = nerftiti-z-bary-TEST}
local file = "core/nefertiti.obj"

local polyhedron, bounding_box = read_obj_file(file)

local graphview = graph3d:new{
  window3d = bounding_box,
  window   = {-5, 4, -7, 5},
  viewdir  = {35, 60},
  size     = {9, 9},
  margin   = {0, 0, 0, 0},
  border   = true
}

draw_colormap_Z_dir(
  polyhedron,
  graphview,
  bounding_box,
  PALETTE
)

graphview:Show()
\end{luadraw}

\end{document}
